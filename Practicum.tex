\documentclass[12pt, letterpaper]{article}
\usepackage[utf8]{inputenc}
 
\title{The challenges of opening a new branch overseas}
\author{Mauricio Soto}
\date{Fall 2017}
 
\begin{document}
 
\begin{titlepage}
\maketitle
\end{titlepage}

\begin{abstract}
This is a simple paragraph at the beginning of the 
document. A brief introduction about the main subject.
\end{abstract}
 
 
\section{Introduction}

The advantages that companies have by opening a branch overseas are constantly increasing.
Owning branches in different locations enables a pool of resources that enterprises would not be able
to obtain otherwise, such as special talents, legal benefits and financial gain. However, 
opening a branch overseas is usually accompanied by a set of challenges inherent to a process of this nature, such as 
difficulties in communication among team members distributed in different locations, 
cultural and language barriers, and transmitting the culture, mechanisms, and knowledge necessary for the 
coworkers in the new branch to perform their tasks adequately. 
This includes training in tools and standard procedures used by the enterprise, as well as the
business logic required to understand the proper requirements and use of the product being built
in the branch.

In my experience, I have had the opportunity to work with very diverse contexts of knowledge and culture transfer in 
an industrial environment. I will compare and contrast my experiences, with a slight emphasis on the 
latest experience I had before joining the Phd program, in which a big well-known enterprise was 
opening a new branch in Costa Rica and was starting to face the challenges that this process encompasses.  




 




\section{Institutional and distributed knowledge}
 
Difficulties dealing with different locations: I have had experiences with development teams where the teams are distributed in 1, 2 and 3 different sites. I can talk about the difficulties this introduces to the process of development, specially difficulties with communication and synchronization efforts.

Passing down knowledge (Understanding the big picture): I can talk about the importance of understanding the big picture when working in a huge project. In particular I could talk about my experience in Experian, where it took my development team over 6 months to understand the overall purpose of the software we were building. I can discuss challenges such as: the fact that the credit score (the basic idea the whole company is built upon, was basically inexistent in our context), how there are several different ways to use it, different stake holders and the process of how it is created, specially since the knowledge isn't in just one expert, but it is distributed knowledge where everyone knows a little and nobody knows everything.

Building a new team overseas: Here I can talk about my experience with Experian were the team I joined was a brand new branch, I was one of the first people to get hired from a team of what became to be ~25 people when I left, and it kept growing. Discuss the challenges that originate with this such as passing down knowledge from people that have worked on this product for decades to a team of young people used to working at a different pase and how it was necessary to build a work culture from scratch.

Agile usage in different contexts: I worked in several different places that state they use agile although I have never been (or known) of a  workplace that applies agile as the manifesto explains it. Here I can talk about how, in my experience, companies with different requirements modify agile practices to adapt to their business needs.

Remote work, when should it be applied?: In my experience remote work is sometimes useful and sometimes it isn't. It provides huge advantages, like the fact that workers can work when they are most productive (in my case it is fantastic because I am productive at night), and also that companies don't have to pay for offices, electricity, etc. but is some cases I have found that remote work actually reduces productivity, for example when a schedule is enforced.

 
\section{Work context}

In the fall of 2013 I was offered a position in Experian, Credit Score Reporting Services as a Software Developer II.
I had never heard the name of this company in my life nor I fully understood what was the main product or service that 
the company provided. 
After the first introduction day, I came to learn that Experian was a well-known company worldwide with over 16,000
employees and their business encompasses credit reporting but it also works with other products such as decision 
analytics, marketing assistance and identity theft. I also learned that we would be working mainly with a branch in 
Costa Mesa, California, and a branch in Santiago, Chile. The branch in Costa Mesa was a well established branch, with 
employees that had over 20 years working for the company. The Costa Rica branch was the latest addition to this growing 
community and I was the third employee in the identity theft field (the other two employees were hired a couple of 
weeks before me). At this point Experian was starting a slow process of knowledge and culture transfer that comprises 
challenges such as:
\begin{itemize}
  \item The fact that there is institutional knowledge, there is no one expert, but knowledge is spread out among a 
large number of different employees, and enough of this knowledge needs to be transmitted to the employees in the new 
branch for them to be able to perform their work properly. 
  \item The fact that the main business logic product, the concept of a "credit score", is not used or known in the 
cultural context of the location of the new branch.
  \item The differences in pace and knowledge between the team in the well established branch, where they have a lot of 
  business knowledge because they have worked there for an extensive amount of time but their knowledge in new tools, 
  frameworks and technologies overall is very limited, contrasted to the team in the new branch where the opposite is
  presented, a team of young developers full of knowledge regarding the latest technologies and used to working at a 
  very high pace, but have no business knowledge for this particular company. 
  \item The team in the well established branch has a large set of BKP's (Best known practices and mechanisms), 
  which they have been using for an extensive amount of time to the point that they have forgotten the main reason
  why these are used. Trying to explain to a new set of young employees the rules that they have to follow but not
  knowing the main reason why they have to follow it, becomes really hard to understand both for the new employees
  as well for the older employees.
  \item The employees in the new branch come with a vast set of new ideas that can be implemented in the company, but
  there is a large resistance to change by the employees in the old branch.
  
  the fact that they have a bunch of best known practices that they already forgot why they do it 
\end{itemize}




\section{Contrast and Compare}


\section{Reflection}

\section{Conclusion}

\section{References} 
 
\end{document}