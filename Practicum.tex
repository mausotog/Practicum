\documentclass[12pt, letterpaper]{article}
\usepackage[utf8]{inputenc}

\usepackage[usenames,dvipsnames]{color}
\newcommand{\todo}[1]
  {{\scriptsize \textbf{\color{red} {#1}}}}
  
   
\title{Challenges faced in the process of knowledge and culture transfer in an industrial environment}
\author{Mauricio Soto}
\date{Fall 2017}
 
\begin{document}
 
\begin{titlepage}
\maketitle
\end{titlepage}

\begin{abstract}
I will write this later
\end{abstract}
 
 
\section{Introduction}

The advantages that enterprises have by opening a branch overseas are constantly increasing.
Owning branches in different locations enables a pool of resources that enterprises would not be able
to obtain otherwise, such as special talents, legal benefits and financial gain. However, 
opening a branch overseas is usually accompanied by a set of challenges inherent to a process of this nature, such as 
difficulties in communication among team members distributed in different locations, 
cultural and language barriers, and transmitting the culture, mechanisms, and knowledge necessary for the 
coworkers in the new branch to perform their tasks adequately. 
This knowledge comprises aspects such as: specific training in tools and standard procedures used by the enterprise, as well as the
business logic required to understand the proper requirements and use of the product being built
in the branch.

In my experience, I have had the opportunity to work with very diverse contexts of knowledge and culture transfer in 
an industrial environment. I will compare and contrast my experiences, with a slight emphasis on the 
latest experience I had before joining the Phd program, in which a big well-known enterprise was 
opening a new branch in Costa Rica and was starting to face the challenges this process encompasses.  




 




\section{Institutional and distributed knowledge}
  \todo{Claire, this background section isn't ready yet, please skip it}

\cite{monperrus11} 

\cite{Newell06} 

\cite{Steinmacher12} 

\cite{Dagenais10} 

\cite{Razavi06} 

\cite{Whitworth06} 

\todo{cite some papers from Jim Herbsleb}


'organizations which design systems ... are constrained to produce designs which are copies of the communication structures of these organizations.' M. Conway
 
 


 
\section{Work context}

In the fall of 2013 I was offered a position in Experian, Credit Score Reporting Services as a Software Developer II.
I had never heard the name of this company in my life nor I fully understood what was the main product or service that 
the company provided. 
After the first introduction day, I came to learn that Experian was a well-known company worldwide with over 16,000
employees and their business encompasses credit reporting but also other products such as decision 
analytics, marketing assistance and identity theft. I also learned that we would be working mainly with a branch in 
Costa Mesa, California, and a branch in Santiago, Chile. The branch in Costa Mesa was a well established branch, with 
employees that had over 20 years working for the company. The Costa Rica branch (which I just joined) was the 
latest addition to this growing 
community and I was the third employee in the identity theft field (the other two employees were hired a couple of 
weeks before me). When I left the company, a year later, the Identity Theft field had  over 25 employees and it 
kept growing. Experian had been going through a slow process of knowledge and culture transfer that comprises 
challenges such as the following:
\begin{itemize}
  \item There is institutional knowledge that needs to be transfered to the new branch. This means there is not a single expert that 
  knows the entirety of the knowledge needed, but knowledge is spread out among a 
large number of different employees, and enough of this knowledge needs to be transmitted to the employees in the new 
branch for them to be able to perform their work properly.
  \item The differences in pace and knowledge between the team in the well established branch, where  
  business knowledge is internalized and well understood by the employees, because the employees working in the old branch have an extensive amount of time working there. This is contrasted with their limited knowledge in new tools, 
  frameworks and technologies overall. We can contrast this with the team in the new branch, where the opposite is
  presented, a team of young developers full of knowledge regarding the latest technologies and used to working at a 
  very high pace, but have no business knowledge for this particular company. 
  \item The team in the well established branch has a large set of BKP's (Best known practices and mechanisms), 
  which they have been using for an extensive amount of time to the point that they have forgotten the main reason
  why these are used. Trying to explain to a new set of young employees the rules that they have to follow but not
  knowing the main reason why they have to follow it, becomes really hard to understand both for the new employees
  as well for the older employees.
  \item The fact that the core business logic product, the concept of a "credit score", is not used or known in the 
cultural context of the new branch.
  \item Resistance to change by the employees in the old branch whose voice has more weight
  because of seniority. This is contrasted with the employees in the new branch, who have a vast set of new ideas 
  that can be implemented in the company, but also lack experience and knowledge regarding how the company works.
  \item Difficulties introduced by location disparities such as geographical and time zone differences. Which lead to 
  challenges in the development process, specially difficulties with communication and synchronization efforts.
\end{itemize}




\section{Contrast and Compare}
Starting in 2010 and before joining the PhD program, I had formally worked for 4 different companies: Experian Credit 
Score Reporting Services (Experian), The University of Costa Rica (UCR), Rutgers University The State University of New 
Jersey (Rutgers), Intel Corporation (Intel). All of which 
have a very different set of working environments and transmitted knowledge in very different ways. A recapitulation of
these experiences is seen in Table~\ref{comparisonTable}, where it is entailed the time I worked for the company, the 
size of the product I was working with (measured in lines of code), if the work was remote or in-office, the size of the 
team I worked with, the number of sites the team was distributed in and the main programming language I worked with.

It is particularly interesting to compare these different experience because of the diversity that they contribute
to my experience in knowledge transfer.

In The University of Costa Rica I worked as a Web Developer and then obtained a promotion to Webmaster. 
This experience is different in several different ways. One of which is that the institutional knowledge 
in this case was minimum.
This was a research center with a small team of developers who would constantly get requirements to build small products 
or implement changes to existing products, which would usually take between a month to six months to get built.

The previous Webmaster was a single person who knew the high level details for most the products built and knew to a 
large extent the knowledge necessary to perform properly as a Webmaster in this context.
In this case, the knowledge transfer experience I had was about 3 hours with this previous Webmaster, who very kindly
indicated to me the knowledge he though was necessary for me to perform well, while I asked him the questions I could 
think of. The biggest limitation in this experience would be the time constraint. Since I had only one meeting with 
this person and further on, if some other question to mind, or if I stumbled upon another problem, the knowledge
transfer process would be over by then.

I later came to realize that 3 hours of training was not quite enough for a position that I was going to hold for 4
years. It quickly became a very challenging experience for me, mostly because most of the learning had to be done 
by myself.

The second experience I will talk about was when I worked for 9 months as a Web Developer in Rutgers University, The
State University of New Jersey. This is a very interesting experiences to comtrast with because the knowledge 
transfer experience I had was written, not verbal. The code was really well documented, but the person who had 
written the comments was already gone. I had the advantage that I could 

I first was interviewed by the director of a recreational facility, a non computer science oriented person. He broadly 
explained to me that I was going to 
perform several changes to a program used by the recreational facility to perform some needed record keeping,
test taking, etc. Once I started the job, the director gave me the password to the server and told me to 
get myself familiarized with the program (a ~5,000 line PHP program). At this point I had to go through
the program understanding what it was supposed to do, reading the comments, running it in different was, making small 
changes to the 
functionality and re-running it to confirm that I was properly understanding the functionality, and after a couple of 
weeks of doing this I was able to understand generally how the program worked and how to perform the changes I was asked
to implement.

Finally, the third experience I will contrast is the experience I had at Intel Corporation, where I was hired
as a Software Developer in February of 2012. In this opportunity, I joined a group of 3 people (myself included),
two developers (located in Costa Rica) and one project manager (located in Arizona). Both branches, Costa Rica 
and Arizona had existed for over 20 years and the working culture had been well assimilated and internalized
at this point, which helped softened my immersion into the work environment.

Probably the factor that made my transition more pleasant regarding transfer of knowledge, was the fact that 
the other developer I was working with was, first of all, knowledgeable. He knew the technologies and processes
very well, and was able to explain to me with as much detail as needed how to perform a task I was not familiar with. 
Second of all, he was available. We worked in the same cubicle and when I needed further explanation of any given 
task we would have the location proximity and he would have the kindness to take a couple of minutes of his time
to explain some bullet to me. Also, since we were regular co-workers, knowledge transfer did not have a particular
time limitation as opposed to my experience in the University of Costa Rica, so knowledge would flow in my
direction in a very natural way. First when I was starting I would need a lot more, and later on less and less 
until most of the knowledge I needed was assimilated, with the exception of very sporadical questions. 





\begin{table}[] 
\centering
\caption{Comparison of knowledge transfer within 4 different experiences}
\label{comparisonTable}
\begin{tabular}{|l|l|l|l|l|}
\hline
                         & \textbf{Experian} & \textbf{UCR}   & \textbf{Rutgers} & \textbf{Intel} \\ \hline
\textbf{Worked for}      & 1 year            & 4 years        & 9 months         & 1 year         \\ \hline
\textbf{Size of product} & 1M+ loc           & small products & $\sim$5k loc     & $\sim$5k loc   \\ \hline
\textbf{Work location}   & In-Office         & Remote         & Remote           & In-Office      \\ \hline
\textbf{Team size}       & 7                 & 5              & 2                & 3              \\ \hline
\textbf{Number of sites} & 3                 & 1              & 1                & 2              \\ \hline
\textbf{Main language}   & Java              & Java           & Php              & C\#            \\ \hline
\end{tabular}
\end{table}

\section{Reflection}

\section{Conclusion}


\bibliographystyle{abbrv}
\bibliography{sigproc}  % sigproc.bib is the name of the Bibliography in this case
% You must have a proper ".bib" file
%  and remember to run:
% latex bibtex latex latex
% to resolve all references

 
\end{document}