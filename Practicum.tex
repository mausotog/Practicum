\documentclass[12pt, letterpaper, twoside]{article}
\usepackage[utf8]{inputenc}
 
\title{First document}
\author{Hubert Farnsworth \thanks{funded by the ShareLaTeX team}}
\date{February 2014}
 
\begin{document}
 
\begin{titlepage}
\maketitle
\end{titlepage}

\begin{abstract}
This is a simple paragraph at the beginning of the 
document. A brief introduction about the main subject.
\end{abstract}
 
These are some ideas I have come up with.
Difficulties dealing with different locations: I have had experiences with development teams where the teams are distributed in 1, 2 and 3 different sites. I can talk about the difficulties this introduces to the process of development, specially difficulties with communication and synchronization efforts.

Passing down knowledge (Understanding the big picture): I can talk about the importance of understanding the big picture when working in a huge project. In particular I could talk about my experience in Experian, where it took my development team over 6 months to understand the overall purpose of the software we were building. I can discuss challenges such as: the fact that the credit score (the basic idea the whole company is built upon, was basically inexistent in our context), how there are several different ways to use it, different stake holders and the process of how it is created, specially since the knowledge isn't in just one expert, but it is distributed knowledge where everyone knows a little and nobody knows everything.

Building a new team overseas: Here I can talk about my experience with Experian were the team I joined was a brand new branch, I was one of the first people to get hired from a team of what became to be ~25 people when I left, and it kept growing. Discuss the challenges that originate with this such as passing down knowledge from people that have worked on this product for decades to a team of young people used to working at a different pase and how it was necessary to build a work culture from scratch.

Agile usage in different contexts: I worked in several different places that state they use agile although I have never been (or known) of a  workplace that applies agile as the manifesto explains it. Here I can talk about how, in my experience, companies with different requirements modify agile practices to adapt to their business needs.

Remote work, when should it be applied?: In my experience remote work is sometimes useful and sometimes it isn't. It provides huge advantages, like the fact that workers can work when they are most productive (in my case it is fantastic because I am productive at night), and also that companies don't have to pay for offices, electricity, etc. but is some cases I have found that remote work actually reduces productivity, for example when a schedule is enforced.
 
\end{document}