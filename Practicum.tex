\documentclass[12pt, letterpaper]{article}
\usepackage[utf8]{inputenc}
\usepackage{graphicx}
\usepackage{booktabs}

\usepackage[usenames,dvipsnames]{color}
\newcommand{\todo}[1]
  {{\scriptsize \textbf{\color{red} {#1}}}}
  
   
\title{Challenges Faced in the Process of Knowledge Transfer in an Industrial Environment}
\author{Mauricio Soto\\
mauriciosoto@cmu.edu
}
\date{Fall 2017}
 
\begin{document}
 
\begin{titlepage}
\maketitle


\begin{abstract}
I analyze the knowledge transfer process in several industrial contexts, corresponding to my 
previous professional experiences in industrial environments.
I contrast the four different industrial experiences I had before joining the SE PhD program, and 
how knowledge transfer played an important role in my learning process for each of these companies.
I examine how the resources and mechanisms companies use to handle knowledge transfer impact the time needed for both newcomers and mentors to understand a code base. 
\end{abstract}
\end{titlepage}
 
\section{Introduction}

Domain specific knowledge is one of the key assets to maintaining competitiveness in 
software enterprises.
To sustain competitive advantage, companies must assess their knowledge resources and 
establish their knowledge strategy~\cite{civi00}. In this context, the value of knowledge transfer 
becomes crucial.

Effective knowledge transfer is a particularly challenging problem because of the characteristics of
knowledge itself. Knowledge is distributed, ambiguous and 
disruptive~\cite{Newell06}, making its transfer a
slow and troublesome process. 
In the context of the software industry, this knowledge includes \textbf{technical
knowledge}, e.g., specific training in 
programming languages, tools, or frameworks used by the enterprise. It also comprises
\textbf{business logic} know-how, such as standard procedures and domain specific 
expertise required to understand the proper use and requirements of the product being built.
Social problems that occur during knowledge transfer are also part of the 
challenges faced, such as resistance to power/knowledge shifts or close-mindedness.

A clear example of when substantial knowledge needs to be transferred is when an existing 
enterprise opens a new branch.
Owning branches in multiple locations enables a pool of resources that enterprises would not be able
to obtain otherwise, such as special talents, legal benefits, or financial gain~\cite{ceruttia07}. However, 
opening a new branch is accompanied by a set of inherent challenges, such as 
difficulties in communication among team members distributed among different locations;
cultural and language barriers; and transmitting the knowledge, culture, and mechanisms necessary for the 
coworkers in the new branch to perform their tasks adequately.

I have had the opportunity to work in diverse contexts of knowledge transfer in 
an industrial environment. I will compare and contrast my experiences, with a slight emphasis on the 
latest experience I had before joining the PhD program, wherein a large, well-known enterprise 
opened a new branch in Costa Rica.

\section{Knowledge Management and Transfer}
Knowledge is one of the most important organizational resources that enterprises possess.
The transfer of this knowledge, especially from senior to junior collaborators, 
is key to maintaining, augmenting, and
making proper use of that knowledge base. 
\textit{Knowledge management}
is the process of dealing with the management 
of knowledge and its related activities. 
The main goal of these activities is to make the enterprise act as intelligently as possible to secure its viability and overall success, and
to realize the best value of its knowledge assets~\cite{wiig97}.

Organizing the process of passing down knowledge has noticeable gains, 
such as: financial
value, operational benefits, business process improvement and culture~\cite{ibrahim09}. These translate into concrete assets such as
improvements in cost, quality, cycle time, lead time, decision making and resolved complaints.

Learning is tightly connected to the knowledge transfer
process.
There are two kinds of knowledge: When you learn \textbf{about} something, referred to as \textit{explicit} 
knowledge; and when you learn \textbf{to do} something, called \textit{tacit} knowledge~\cite{cook99}. 
\textit{Cognitive learning} is when someone learns \textbf{about} a topic, as opposed to 
\textit{behavioral learning} which refers to the act of learning \textbf{to do} something.
In the case of knowledge transfer in a industrial environment, collaborators need both cognitive and
behavioral learning.

When newcomers join a new software project, they typically find themselves in an
unfamiliar project
landscape.
There are three primary factors that impact newcomers'
success when facing a new project: early experimentation, internalizing structure
and culture, and progress validation~\cite{Dagenais10}. These three main factors
have a strong connection to knowledge transfer. 

Knowledge transfer in an industrial context can take mainly two forms:
face-to-face interactions between one or several apprentices and a mentor, or documentation 
(i.e. internal comments in the code, external documentation, educational videos, directives, etc). 
Other practices that boost the quality 
of the learning process are, for example, to appoint mentors whom share recent
contextual information and focus on related issues~\cite{Steinmacher12}. Providing support for selective information disclosure that
maps to the newcomers' mental models becomes a crucial part of the learning
experience. This enables collaboration between apprentice and mentor, and sharing of ideas and reflections, 
enhancing the overall learning process~\cite{Razavi06}. Constant interaction
and geographical proximity between mentor and apprentice also enhance communication, as well as 
practices such as openness, honesty, trust, respect and transparency~\cite{Whitworth06}. 

Documentation is especially important when face-to-face interaction is difficult to obtain.
Documentation is particularly important 
for the software components that are intended to be reused the most~\cite{monperrus11}. Parnas mentions that even
when software components are built using good design, he does not see 
them being
reused if they are not well documented.~\cite{brooks95}. 

In both cases, current collaborators guiding the newcomers is crucial.
Face-to-face human guidance is invaluable since it is a two-way dialog. Documentation,
although valuable, is a mechanism
where information travels in just one direction and usually lacks detail~\cite{Dagenais10}.
 
\section{Work Context}
Between 2010 and 2014, I formally worked for four companies: Experian Credit 
Score Reporting Services (Experian), The University of Costa Rica (UCR), Rutgers University - The State University of New 
Jersey (Rutgers), and Intel Corporation (Intel). All of these companies 
have a different set of working environments and transmitted knowledge in different ways. Table~\ref{comparisonTable}
shows a summary of
these experiences, including the length of time I worked for the company, the 
approximated size of the product I worked on, whether the work was remote or in-office, the size of my 
team, the number of sites the team was distributed over, the main programming language,
and whether it was a part- or full-time position.

\begin{table}[] 
\centering
\caption{Comparison of four different experiences}
\label{comparisonTable}
\begin{tabular}{lllll}
\hline
                         & \textbf{Experian}      & \textbf{UCR}   & \textbf{Rutgers} & \textbf{Intel} \\ \hline
                         
\textbf{Worked for}      & 1 year                 & 4 years        & 9 months         & 1 year         \\ 
\textbf{Size of product} & 1M+ loc                & small products & $\sim$5k loc     & $\sim$5k loc   \\ 
\textbf{Work location}   & In-Office              & Remote         & Remote           & In-Office      \\ 
%\textbf{Team size}       & \multicolumn{1}{r}{7}& \multicolumn{1}{r}{5}& \multicolumn{1}{r}{2}& \multicolumn{1}{r}{3}\\ \hline
%\textbf{Number of sites} & \multicolumn{1}{r}{3}& \multicolumn{1}{r}{1}& \multicolumn{1}{r}{1}& \multicolumn{1}{r}{2}\\ \hline
\textbf{Team size}       & 7 developers           & 5 developers   & 2 developers     & 3 developers   \\ 
\textbf{Number of sites} & 3 sites                & 1 sites        & 1 sites          & 2 sites        \\ 
\textbf{Main language}   & Java                   & Java           & Php              & C\#            \\ 
\textbf{Part/Full Time}  & Full                   & Part           & Part             & Full           \\ \hline
\end{tabular}
\end{table}

\subsection{Experian - Credit Score Reporting Services}
In the Fall of 2013, I was offered a position at Experian - Credit Score Reporting Services, as a Software Developer II in the site located in Heredia, Costa Rica.
Experian is a well-known company worldwide, with over 16,000
employees. Their business encompasses credit reporting, but also other products such as decision 
analytics, marketing assistance and identity theft. I learned that me and 
the other newcomers would be working mainly with two other branches: 
Costa Mesa, California, and Santiago, Chile. 

The branch in Costa Mesa was well established, with some employees having worked for over 20 years in the company. 
The Costa Rica branch (which I had just joined) was the 
latest addition to this growing 
community. I was the third employee in the Identity Theft field; the other two employees were hired a couple of 
weeks before me. When I left the company a year later, the Identity Theft area had over 25 employees and it 
kept growing. 

I worked with a couple
of different groups, each building a distinct product. In this sense, Experian was
similar to how research 
is produced in academia: A person works with several groups composed of collaborators with different 
backgrounds and diverse levels of expertise.

The main project I worked on was a tool for identity theft prevention to increase the security of end users.
When end users want to check their credit score, they go to the Experian website and provide authentication
information. If the information provided is below a certain threshold of accuracy, our tool is invoked. We would collect 
information from other companies about the customer, and ask questions about personal information that only this particular
individual should know( e.g.: their eye color as reported in their driver's license).

I worked on this project for a year. Throughout that year, the process of understanding this massive 
code base was an uphill experience for me, in part because the code was rarely documented
and there were no accessible mentors to ask for guidance. Similar to me, new employees where being constantly hired for the new Costa Rican 
branch, and they had to go through a similar learning process. My previously mentioned experience falls into a large
umbrella of challenges faced by companies when dealing with knowledge transfer. 
From my experience, the challenges the enterprise faced included the following:

\begin{itemize}
  \item There is \textit{institutional knowledge} that needs to be transferred to the new branch. There is no single expert with  the entirety of the necessary knowledge to be transferred. Instead, knowledge is spread out among a 
  large number of different collaborators. Enough of this knowledge needs to be transmitted from the employees in the 
  older branches to the employees in the new 
  branch for them to be able to perform their work adequately. In my experience, if I wanted to understand a particularly difficult
  section of code, I would ask a developer working on that code. This developer could tell me specifics about this 
  module in particular, but would not know for example, where the module is being called or where do the parameters come
  from.
  \item The differences in pace and knowledge between the team in the well-established branch and the new branch.
  In the former, business knowledge is internalized and well understood by the employees; as opposed to the employees of the new
  branch who just started working for the company. 
 By contrast, in the old branch the employees usually have limited knowledge of new tools, 
  frameworks or technologies overall. In the new branch, the opposite was
  common: a team of young developers full of knowledge regarding the latest technologies and used to working at a 
  very high pace, but without the business knowledge for this particular company. 
  \item The fact that the core business logic product, the concept of a ``credit score", was not commonly used or known in the 
  cultural context of the new branch. In Costa Rica, credit scores are rarely used. This is by contrast with the 
  United States, where customers' credit scores are checked if they want a cell phone plan, rent an apartment or open a bank account.
  In Costa Rica, credit scores are used mostly for long term loans such as mortgages, and companies do not require your
  authorization to check it. Therefore, even if
  your credit score is checked, you would not know about it, which makes the concept of the ``credit score" not well-known or fully understood in the new branch's local culture.
  \item Resistance to change by the employees in the old branch whose voice has more weight
  because of seniority. This is contrasted with the employees in the new branch, who have new ideas 
  that can be implemented in the company, but also lack experience and knowledge regarding how the company works.
  I noticed this the most when the employees of the new branch suggested the use of Jenkins to track 
  development progress. Even though we got a thumbs up to start using it and we did,
  the more senior developers refused to, because they did not want to spend time learning to use a new tool.
  \item Difficulties introduced by location disparities such as geographical and time zone differences. These lead to 
  challenges in the development process, specially difficulties with communication and synchronization efforts.
  This was more noticeable in the mornings. We started working at 7am Costa Rican time, while the team in Costa Mesa usually
  started working at 11am in Costa Rican time (9am California time). This usually meant that if we needed clarification
  or any kind of communication, we needed to wait several hours every day to be able to interact with the more experienced team.
  
\end{itemize}

\subsection{Past Experiences}
It is particularly interesting to compare to my other past experiences because of the diversity and lessons learned
they contribute to my understanding of knowledge transfer.

\subsubsection{University of Costa Rica}
At The University of Costa Rica, I worked as a Web Developer and Webmaster. 
The institutional knowledge in this case was minimal.
I worked for a research center, with a small team of developers. We constantly received requests to build small products 
or implement changes to existing products. These tasks usually took between one and six months to build.
The previous Webmaster was a single person who knew the high level details for most of the products. He possessed, 
to a large extent, all the knowledge necessary to perform properly as a Webmaster in this context.

In this case, the knowledge transfer experience I had was approximately three hours with the 
previous Webmaster. He very kindly
indicated to me the knowledge he thought was necessary for me to perform well, while I asked any questions I could 
think of. The biggest limitation in this experience was the time constraint: I had only one meeting with 
this person.

Ultimately, three hours of training was not quite enough for a position that I held for four
years. It quickly became a very challenging experience for me, largely because I had to learn
most of the technical knowledge and business logic on my own. This presented the steepest learning curve of all my experiences. The 
majority of the knowledge I obtained was not provided by anyone within the company, but rather via third party
sources or by my own experimentation. For example, we used several different frameworks and plug-ins. 
I had to look for and read the documentation of 
these frameworks and plug-ins to understand how they worked. When I couldn't find documentation to read, I relied
on other learning mechanisms such as modifying the source code and re-running it to compare its behavior
before and after the changes where performed. 

Unlike my other experiences this company had a very weak knowledge transfer mechanism.
This resulted in considerable time spent by the newcomers looking for knowledge that was not passed down to them. 
The newcomers took much longer to be able to obtain knowledge that was already present in previous developers,
and could have been learned in a more straightforward way if the knowledge was properly handed down to them 
in the form of documentation or face-to-face communication.

\subsubsection{Rutgers University}
My 9-month experience as a Web Developer in Rutgers University is interesting to contrast with the others because the knowledge 
transfer was written, not verbal. The source code was really well documented, but the person who had 
written the comments had already left the position. Having written documentation was advantageous because 
I could access it where/whenever it was most convenient for me. But it does not have the dialog that face-to-face
mechanisms provide, which is particularly helpful when I had doubts not covered by the documentation. 

When I applied for this position, I was first interviewed by the director of the department, 
a non-technical person. He broadly explained to me that I was going to 
perform several changes to a program used by the department for record keeping,
staff training, etc. Once I started the job, the director gave me the password to the server and told me to familiarize myself with the program (approximately 5,000 LOC PHP program). At this point I had to go through
the program understanding what it was supposed to do, reading the comments, running it in different ways, making small 
changes to the 
functionality and re-running it to confirm that I was properly understanding the functionality. After a couple of 
weeks of doing this I was able to understand generally how the program worked and how to perform the changes I was asked
to implement.

\subsubsection{Intel}
Finally, I worked at Intel Corporation, where I was hired
as a Software Developer in February of 2012. I joined a group of 3 people (myself included):
two developers (located in Costa Rica) and one project manager (located in Arizona). Both branches
had existed for over 20 years and the working culture had been well assimilated and internalized. 
This helped soften my immersion into the work environment since I had to only adapt to the already
existing working culture(as opposed to Experian, where we were creating a culture from scratch).

One important factor that made my transition more pleasant regarding knowledge transfer was that 
the other developer I was working with was knowledgeable. He knew the technologies and processes
being used in this team very well.
He was also highly available. We worked in the same cubicle and when I needed further explanation of any given 
task, his proximity and willingness
to help softened my learning curve making me 
learn much faster. Also, since we were regular co-workers, knowledge transfer did not have a
time limit (unlike my experience at the University of Costa Rica), so knowledge could flow in my
direction very naturally in an ad hoc basis. 


\section{Reflection}

In my experience, a hybrid approach using both face-to-face and documentation is 
the best and fastest way to help newcomers to adapt and understand
a new landscape. In the experiences where I had a good balance between documentation and face-to-face time
(such as at Intel), I was able to learn much faster and assimilate the code than I could when
I had only documentation (Rutgers) or very limited face-to-face interactions with the mentors (Experian and 
UCR). 

Previous studies have focused on the difference between explicit knowledge and tacit knowledge (learning \textit{about} 
something versus learning \textit{to do} something)~\cite{cook99,civi00}. Both types of knowledge are important from a software
engineering perspective. I focus on the differences between knowledge 
acquired through one-way communication (e.g. code documentation) versus two-way communication (e.g. face-to-face knowledge transfer). 
Documentation (one-way communication) is a common practice in software engineering, 
and mentors' help (two-way communication) usually involves senior developers, 
whose time is valuable. 

I have also validated with my personal experience key insights from previous literature. 
For example, I have observed that early experimentation (UCR and Rutgers) and progress validation (Intel) are key features
in the learning process of software projects~\cite{Dagenais10}. I have also witnessed the advantages of 
learning from an active developer with social skills~\cite{Steinmacher12}
and geographical proximity~\cite{Whitworth06} (Intel and Experian). 

\subsection{One-way Communication}
I have observed that proper internal documentation is the most 
important of all the forms of documentation for new developers to understand the purpose of the code they will
be working with. This includes meaningful names for classes, 
methods and variables, clean code, and developer written
comments.

The importance of this particular kind of documentation comes from the fact that it is immediate.
Newcomers go through the code to try to understand it by making sense of the 
developer written comments and class, method, and variable names. They go through
a cognitive matching process between the expected behavior of the program and the code. 
If such comments, functionality and component names match the developers' mental model 
of the program expected behavior, then the knowledge transfer will be immediate and effective.

Unfortunately, in my experience, developer written comments are very rarely updated, which results in 
comments that do not match what the code is currently doing. Throughout all my experiences 
I saw outdated comments on a daily basis, though less frequently when using software 
meant to be highly reproducible(e.g., frameworks). In all other cases,
where understandability was not a priority, outdated comments were a common problem.

This leads to newcomers not 
trusting the comments.
Outdated documentation is in general a well known problem in software maintenance~\cite{lethbridge03} and this is a scenario
where we can see in a very clear way the impact of this problem, since it interferes with the 
knowledge transfer process and the newcomer learning experience.


\subsection{Two-way Communication}
In my experience, two-way communication between newcomers and mentors
represents a much smaller component of the knowledge transfer process. One of the main reasons behind this is that
senior developers' time is a highly valued asset which must be used for high priority tasks, and often knowledge transfer to newcomers does not qualify as such.

In my experience, this kind of knowledge transfer is less necessary when the documentation in the project is of high
quality. As documentation quality decreases, the need for a mentor increases. This can be described
as an inversely proportional relationship between the quality of the documentation in the project and the 
time needed to be spent by both the mentor and the newcomer in face-to-face communication.

In my experience at Rutgers, for example, the internal documentation of the project was of high quality. Therefore
the need for a two-way documentation was lesser than my experience at Experian, where the documentation was of low quality and 
I needed much more mentorship.

\subsection{Conclusion}

From my experience, the most 
important of the different kinds of documentation, is the internal documentation which encompasses
names of classes, variables, methods, and developer written comments. The importance of this comes from
the immediateness of this kind of documentation to satisfy doubts the newcomers may have.
Anecdotically, this is the most commonly used source of information by newcomers. 
Two-way communication is needed in a inversely 
proportional manner to the quality of said documentation.

Some high level factors that can highly improve the learning experience of newcomers when joining a new software 
project are:
\begin{itemize}
  \item The presence of both documentation and an expert, with emphasis in the former
  \item Knowledgeable and available mentor(s)
  \item Include newcomers into already existing teams with an already defined work culture 
\end{itemize}

I can derive from this information that applying best practices from code maintenance literature such as 
using meaningful names for classes, variables, and methods in the software being built, and constantly updating
the developer written comments is optimal for the knowledge transfer process in newcomers. Therefore making
the need for two-way communication between newcomers and mentors minimal. This is also advantageous for the
company since mentors tend to be more senior developers and the time that mentors spend instructing newcomers
is time they are not creating new products or fixing important errors, tasks that usually have a higher priority
in industrial environments.

\bibliographystyle{abbrv}
\bibliography{sigproc}  % sigproc.bib is the name of the Bibliography in this case
% You must have a proper ".bib" file
%  and remember to run:
% latex bibtex latex latex
% to resolve all references

 
\end{document}